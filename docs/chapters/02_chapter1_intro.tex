% chapters/02_chapter1_intro.tex
% Chapter 1: Introduction (Mở Đầu)
% Author: Nguyễn Tấn Khôi
% Date: 2025-11-12

\chapter{MỞ ĐẦU}
\label{chap:introduction}

%% ============================================================================
%% Section 1.1: Lý do chọn đề tài
%% ============================================================================

\section{Lý do chọn đề tài}
\label{sec:motivation}

Trong những năm gần đây, hệ thống đa robot đã và đang được ứng dụng rộng rãi trong nhiều lĩnh vực như kho hàng tự động, nhà máy thông minh, và logistics. Việc điều khiển nhiều robot hoạt động đồng thời trong cùng một môi trường đặt ra thách thức lớn về tránh va chạm, đặc biệt khi số lượng robot tăng lên.

Các phương pháp điều khiển tập trung truyền thống gặp khó khăn về khả năng mở rộng (scalability) khi số lượng robot lớn, do yêu cầu về tính toán và truyền thông tăng theo cấp số nhân. Ngược lại, phương pháp điều khiển phân tán, trong đó mỗi robot tự quyết định dựa trên quan sát cục bộ, cho phép hệ thống mở rộng tốt hơn.

Học tăng cường sâu (Deep Reinforcement Learning - DRL) đã chứng minh khả năng giải quyết các bài toán điều khiển phức tạp. Đặc biệt, thuật toán Proximal Policy Optimization (PPO) được đánh giá cao về độ ổn định và hiệu quả trong việc huấn luyện các policy đặc biệt trong việc Supervised Fine-Tuning các mô hình ngôn ngữ lớn. Tuy nhiên, việc áp dụng DRL vào bài toán đa robot với số lượng lớn vẫn còn nhiều thách thức cần nghiên cứu.

%% ============================================================================
%% Section 1.2: Mục tiêu nghiên cứu
%% ============================================================================

\section{Mục tiêu nghiên cứu}
\label{sec:objectives}

Mục tiêu chính của luận văn là nghiên cứu và phát triển thuật toán điều khiển phân tán cho hệ đa robot tránh va chạm sử dụng học tăng cường sâu. Cụ thể:

\begin{enumerate}
    \item Nghiên cứu thuật toán PPO và các phương pháp học tăng cường cho bài toán đa robot.

    \item Huấn luyện mô hình neural network có khả năng điều khiển robots tránh va chạm trong môi trường mô phỏng.

    \item Đề xuất các cải tiến về learning rate scheduling, value clipping và policy để đáp ứng với thực nghiệm.

    \item Thiết kế phần cứng robot prototype với cảm biến LiDAR, Jetson và Arduino đơn giản và chi phí thấp.

    \item Triển khai và kiểm nghiệm thực tế mô hình trên robot trong thực nghiệm.
\end{enumerate}

%% ============================================================================
%% Section 1.3: Đối tượng và phạm vi nghiên cứu
%% ============================================================================

\section{Đối tượng và phạm vi nghiên cứu}
\label{sec:scope}

\subsection{Đối tượng nghiên cứu}

Đối tượng nghiên cứu là hệ thống đa robot di động (mobile robots) với các đặc điểm:

\begin{itemize}
    \item \textbf{Kiểu robot:} Nonholonomic robots (robot không toàn hướng) di chuyển trên mặt phẳng 2D
    \item \textbf{Cảm biến:} LiDAR $360^\circ$ với độ phân giải $0.8^\circ$
    \item \textbf{Hành động:} Điều khiển vận tốc tuyến tính $v$ và vận tốc góc $\omega$
    \item \textbf{Số lượng:} 44 robots hoạt động đồng thời trong mô phỏng; 2 robot trong thực nghiệm
\end{itemize}

\subsection{Phạm vi nghiên cứu}

Luận văn tập trung vào các khía cạnh sau:

\begin{itemize}
    \item Thuật toán học tăng cường PPO cho bài toán điều khiển phân tán
    \item Môi trường mô phỏng 2D với chướng ngại vật tĩnh
    \item Quan sát cục bộ (local observation) từ cảm biến LiDAR
    \item Không sử dụng truyền thông giữa các robot (fully decentralized)
    \item Tuning PID ổn định điều khiển robot thực nghiệm
    \item Triển khai được model huấn luyện từ mô phỏng vào robot thực nghiệm
    \item Sensor fusing để robot có thể hoạt động và xác định chính xác vị trí của nó trong môi trường với các cảm biến giá rẻ
    \item Mục tiêu: Đạt tỉ lệ thành công robot di chuyển từ vị trí xuất phát đến đích mà không va chạm trên 80\%
\end{itemize}

\textbf{Ngoài phạm vi:} Môi trường 3D, multi-task learning, xác định vị trí chính xác robot.

%% ============================================================================
%% Section 1.4: Ý nghĩa khoa học và thực tiễn
%% ============================================================================

\section{Ý nghĩa khoa học và thực tiễn}
\label{sec:significance}

\subsection{Ý nghĩa khoa học}

Luận văn đóng góp vào lĩnh vực nghiên cứu đa robot và học tăng cường thông qua:

\begin{itemize}
    \item Đánh giá tính ứng dụng của mô hình học tăng cường trong điều khiển hệ robot phân tán 
    \item Đề xuất các cải tiến về thuật toán huấn luyện PPO:
    \begin{itemize}
        \item \textit{Adaptive Learning Rate Scheduler}: Duy trì learning rate khi performance cải thiện
        \item \textit{Value Clipping}: Giảm instability trong quá trình huấn luyện
        \item \textit{Separate Optimizers}: Sử dụng learning rate khác nhau cho actor và critic
        \item \textit{Learning Rate Warmup}: Tăng dần learning rate trong giai đoạn đầu
    \end{itemize}

    \item Nghiên cứu khả năng thực tiễn của mô hình với robot Nonholonomic
    
\end{itemize}

\subsection{Ý nghĩa thực tiễn}

Kết quả nghiên cứu có thể ứng dụng vào:

\begin{itemize}
    \item \textbf{Kho hàng tự động (Automated Warehouses):} Điều khiển nhiều robot AGV (Automated Guided Vehicle) di chuyển hàng hóa hiệu quả.

    \item \textbf{Nhà máy thông minh (Smart Factories):} Phối hợp nhiều robot công nghiệp trong dây chuyền sản xuất.

    \item \textbf{Logistics và vận tải:} Quản lý đội xe tự hành trong khu vực hạn chế.
\end{itemize}

Kết quả đạt được trên robot thực nghiệm chứng minh tính khả thi của phương pháp trong điều kiện thực tế với mô hình phần cứng đơn giản và dễ dàng mở rộng với số lượng robot lớn.

%% ============================================================================
%% Chapter conclusion (required by FR-014)
%% ============================================================================
