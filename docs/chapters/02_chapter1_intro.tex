% chapters/02_chapter1_intro.tex
% Chapter 1: Introduction (Mở Đầu)
% Author: Nguyễn Tấn Khôi
% Date: 2025-11-12

\chapter{MỞ ĐẦU}
\label{chap:introduction}

% Chapter introduction (required by FR-013)
Chương này giới thiệu bối cảnh, động lực, mục tiêu, phạm vi và ý nghĩa của đề tài nghiên cứu về ứng dụng học tăng cường trong điều khiển phân tán hệ đa robot tránh va chạm.

%% ============================================================================
%% Section 1.1: Lý do chọn đề tài
%% ============================================================================

\section{Lý do chọn đề tài}
\label{sec:motivation}

Trong những năm gần đây, hệ thống đa robot đã và đang được ứng dụng rộng rãi trong nhiều lĩnh vực như kho hàng tự động, nhà máy thông minh, và logistics. Việc điều khiển nhiều robot hoạt động đồng thời trong cùng một môi trường đặt ra thách thức lớn về tránh va chạm, đặc biệt khi số lượng robot tăng lên.

Các phương pháp điều khiển tập trung truyền thống gặp khó khăn về khả năng mở rộng (scalability) khi số lượng robot lớn, do yêu cầu về tính toán và truyền thông tăng theo cấp số nhân. Ngược lại, phương pháp điều khiển phân tán, trong đó mỗi robot tự quyết định dựa trên quan sát cục bộ, cho phép hệ thống mở rộng tốt hơn.

Học tăng cường sâu (Deep Reinforcement Learning - DRL) đã chứng minh khả năng giải quyết các bài toán điều khiển phức tạp. Đặc biệt, thuật toán Proximal Policy Optimization (PPO) được đánh giá cao về độ ổn định và hiệu quả trong việc huấn luyện các chính sách điều khiển. Tuy nhiên, việc áp dụng DRL vào bài toán đa robot với số lượng lớn (trên 40 robots) vẫn còn nhiều thách thức cần nghiên cứu.

%% ============================================================================
%% Section 1.2: Mục tiêu nghiên cứu
%% ============================================================================

\section{Mục tiêu nghiên cứu}
\label{sec:objectives}

Mục tiêu chính của luận văn là nghiên cứu và phát triển thuật toán điều khiển phân tán cho hệ đa robot tránh va chạm sử dụng học tăng cường sâu. Cụ thể:

\begin{enumerate}
    \item Nghiên cứu thuật toán PPO và các phương pháp học tăng cường cho bài toán đa robot.

    \item Huấn luyện mô hình neural network có khả năng điều khiển 44-58 robots tránh va chạm trong môi trường mô phỏng.

    \item Đạt tỷ lệ thành công tối thiểu 71\% trong huấn luyện và 88\% trong kiểm thử.

    \item Đề xuất các cải tiến về learning rate scheduling, value clipping và chiến lược huấn luyện để cải thiện hiệu suất.

    \item Thiết kế phần cứng robot prototype với cảm biến LiDAR để chuẩn bị cho việc triển khai thực tế.
\end{enumerate}

%% ============================================================================
%% Section 1.3: Đối tượng và phạm vi nghiên cứu
%% ============================================================================

\section{Đối tượng và phạm vi nghiên cứu}
\label{sec:scope}

\subsection{Đối tượng nghiên cứu}

Đối tượng nghiên cứu là hệ thống đa robot di động (mobile robots) với các đặc điểm:

\begin{itemize}
    \item \textbf{Kiểu robot:} Nonholonomic robots (robot không toàn hướng) di chuyển trên mặt phẳng 2D
    \item \textbf{Cảm biến:} LiDAR 360 độ với 454 điểm quét
    \item \textbf{Hành động:} Điều khiển vận tốc tuyến tính $v$ và vận tốc góc $\omega$
    \item \textbf{Số lượng:} 44-58 robots hoạt động đồng thời
\end{itemize}

\subsection{Phạm vi nghiên cứu}

Luận văn tập trung vào các khía cạnh sau:

\begin{itemize}
    \item Thuật toán học tăng cường PPO cho bài toán điều khiển phân tán
    \item Môi trường mô phỏng 2D với chướng ngại vật tĩnh
    \item Quan sát cục bộ (local observation) từ cảm biến LiDAR
    \item Không sử dụng truyền thông giữa các robot (fully decentralized)
    \item Mục tiêu: Di chuyển từ vị trí xuất phát đến đích mà không va chạm
\end{itemize}

\textbf{Ngoài phạm vi:} Chướng ngại vật động, môi trường 3D, multi-task learning.

%% ============================================================================
%% Section 1.4: Ý nghĩa khoa học và thực tiễn
%% ============================================================================

\section{Ý nghĩa khoa học và thực tiễn}
\label{sec:significance}

\subsection{Ý nghĩa khoa học}

Luận văn đóng góp vào lĩnh vực nghiên cứu đa robot và học tăng cường thông qua:

\begin{itemize}
    \item Đề xuất các cải tiến về thuật toán huấn luyện PPO:
    \begin{itemize}
        \item \textit{Adaptive Learning Rate Scheduler}: Duy trì learning rate khi performance cải thiện
        \item \textit{Value Clipping}: Giảm instability trong quá trình huấn luyện
        \item \textit{Separate Optimizers}: Sử dụng learning rate khác nhau cho actor và critic (tỷ lệ 1:15)
        \item \textit{Learning Rate Warmup}: Tăng dần learning rate trong giai đoạn đầu
    \end{itemize}

    \item Nghiên cứu scalability của thuật toán với số lượng robot lớn (so với bài báo gốc: 4-20 robots).

    \item Phân tích chi tiết 8 revisions thực nghiệm và ảnh hưởng của các hyperparameters.
\end{itemize}

\subsection{Ý nghĩa thực tiễn}

Kết quả nghiên cứu có thể ứng dụng vào:

\begin{itemize}
    \item \textbf{Kho hàng tự động (Automated Warehouses):} Điều khiển nhiều robot AGV (Automated Guided Vehicle) di chuyển hàng hóa hiệu quả.

    \item \textbf{Nhà máy thông minh (Smart Factories):} Phối hợp nhiều robot công nghiệp di động trong dây chuyền sản xuất.

    \item \textbf{Logistics và vận tải:} Quản lý đội xe tự hành trong khu vực hạn chế.
\end{itemize}

Kết quả đạt được (88\% test success rate với 50 robots) gần với bài báo gốc (96.5\% với 20 robots), chứng minh tính khả thi của phương pháp trong điều kiện thử nghiệm khắt khe hơn.

%% ============================================================================
%% Chapter conclusion (required by FR-014)
%% ============================================================================

\vspace{1cm}

Tóm lại, chương này đã trình bày động lực, mục tiêu, phạm vi và ý nghĩa của đề tài nghiên cứu. Các chương tiếp theo sẽ trình bày chi tiết về tổng quan nghiên cứu liên quan, phương pháp thực hiện, kết quả đạt được và kết luận.
