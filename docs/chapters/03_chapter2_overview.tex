% chapters/03_chapter2_overview.tex
% Chapter 2: Literature Review (Tổng Quan)
% Author: Nguyễn Tấn Khôi
% Date: 2025-11-12

\chapter{TỔNG QUAN}
\label{chap:overview}

% Chapter introduction
Chương này trình bày tổng quan về các nghiên cứu liên quan đến học tăng cường, thuật toán PPO, và ứng dụng trong bài toán đa robot tránh va chạm.

\section{Phương pháp của Long et al.}
\label{sec:long_method}

Bài báo gốc của Long et al. \cite{long2018towards} đề xuất phương pháp điều khiển phân tán sử dụng Deep Reinforcement Learning với thuật toán PPO. Mô hình sử dụng kiến trúc CNN với 2 lớp Conv1D và huấn luyện theo 2 giai đoạn, đạt kết quả 96.5-100\% success rate với 4-20 robots.

% Placeholder - nội dung chi tiết sẽ được bổ sung sau

\section{So sánh các phương pháp tránh va chạm}
\label{sec:comparison}

% Placeholder - nội dung sẽ được bổ sung

\section{Cơ sở lý thuyết}
\label{sec:theory}

\subsection{Reinforcement Learning}
% Placeholder

\subsection{Proximal Policy Optimization (PPO)}
Thuật toán PPO \cite{schulman2017proximal} là một trong những thuật toán policy gradient hiệu quả nhất hiện nay.

% Placeholder - nội dung chi tiết sẽ được bổ sung

\section{Đánh giá tình hình nghiên cứu}
\label{sec:research_gap}

% Placeholder - nội dung sẽ được bổ sung

\vspace{1cm}
Chương này đã tổng quan các nghiên cứu liên quan và xác định khoảng trống nghiên cứu cần giải quyết.
